\chapter*{Lời nói đầu}
\addcontentsline{toc}{chapter}{Lời nói đầu} 
\markboth{Lời nói đầu}{} 

\noindent
Xin chào, mình là \textbf{Đinh Công Thái}, sinh viên lớp IT1-K67, ngành Khoa học Máy tính, Trường Đại học Bách khoa Hà Nội. Là nghiên cứu sinh tại Viên Nghiên cứu và Ứng dụng Trí tuệ nhân tạo của \textbf{PGS.TS Nguyễn Phi Lê}, đồng thời nghiên cứu và làm việc dưới sự hướng dẫn của anh \textbf{Nguyễn Tuấn Dũng}.\\


\vspace{1em}
\noindent
\textbf{Vậy cuốn giáo trình này ra đời để làm gì?}  

Mình viết nó cho những bạn đang mơ hồ, không định hướng, không biết AI bắt đầu từ đâu, nhưng có chút tò mò và muốn thử. Nội dung tập trung vào \textbf{Xử lý ngôn ngữ tự nhiên (NLP)}

Sách này không hứa biến bạn thành chuyên gia sau một đêm. Nhưng nó sẽ là một bản đồ để bạn không bị lạc trôi quá lâu, và nếu thấy hứng thú thì có thể đi xa hơn, sớm hơn.


Cách làm của mình khá ra tác phẩm này có thể tóm tắt như sau:

\begin{itemize}
    \item \textbf{Soạn mục lục} $\rightarrow$ dựng cái khung nhà cho con chatbot.
    \item \textbf{Tìm \& gom paper/tài liệu}, để LLM không bịa, đồng thời cũng để cho kiến thức được sâu sắc và bám sát.
    \item \textbf{Chia thư mục theo từng phần/tiểu mục} $\rightarrow$ giữ cấu trúc rõ ràng.
    \item \textbf{Soạn rule} $\rightarrow$ chính là \textit{style guide}: văn phong, yêu cầu ví dụ, công thức, hình minh hoạ…
    \item \textbf{Tool 1 (lưu context)} $\rightarrow$ giúp LLM nhớ liền mạch, không quên mình vừa viết gì ở trang trước.
    \item \textbf{Tool 2 (API xuất bản)} $\rightarrow$ gửi query theo từng mục và nhận lại nội dung thành file tương ứng.
    \item \textbf{Tool 3 (tự động review \& bổ sung)} $\rightarrow$ LLM đọc lại chính nội dung đã sinh, so với rule + nguồn, rồi nhận xét/gợi ý sửa.
\end{itemize}

Kết quả cuối cùng: phần kiến thức cốt lõi thì mình đã kiểm tra lại cẩn thận. 
Còn nếu  có vài chỗ đọc hơi lạ lạ thì có thể đó là \textit{hallucination}.


\vspace{1em}
\noindent
\textbf{Cấu trúc cuốn sách}  

\begin{itemize}[leftmargin=*]
    \item \textbf{Phần 1: Lý thuyết NLP} – kể câu chuyện từ những nền tảng cơ bản cho tới các mô hình hiện đại như Transformer. Đọc phần này để biết “tại sao lại như vậy”.  
    \item \textbf{Phần 2: Thực chiến NLP} – biến lý thuyết thành code, dự án, ứng dụng. Đọc phần này để biết “làm thế nào”.  
\end{itemize}

Hai phần này bổ sung cho nhau. Biết lý thuyết mà không thực hành thì dễ nói cho vui. Còn chỉ thực hành mà không hiểu gốc thì cũng dễ trở thành copy-paste developer.  

\begin{flushright}
\textit{Trân trọng,} \\[0.5em]
Đinh Công Thái
\end{flushright}
