% !TEX root = ../../main.tex
% File: part2/chapters3/chap3_1.tex

\section{Cơ sở dữ liệu Vector (Vector Databases)}
\label{sec:vector_databases}

\subsection{Khái niệm và vai trò trong NLP hiện đại}
\label{ssec:vector_db_concepts}

Trong các chương trước, chúng ta đã thấy rằng "embedding" là ngôn ngữ của NLP hiện đại. Mọi thứ, từ một từ, một câu, đến một tài liệu hoàn chỉnh, đều có thể được biểu diễn bằng một vector dày đặc. Các hệ thống như RAG (mục \ref{sec:rag}) và Bộ nhớ Dài hạn cho Tác tử (mục \ref{ssec:agent_memory}) đều dựa trên một thao tác cốt lõi: **"Cho một vector truy vấn, hãy tìm các vector tương tự nhất trong một tập hợp hàng triệu, thậm chí hàng tỷ vector khác."**

\paragraph{Vấn đề của Tìm kiếm Brute-force}
Cách tiếp cận ngây thơ nhất là tính toán độ tương đồng (ví dụ: cosine similarity) giữa vector truy vấn và \textit{tất cả} các vector trong cơ sở dữ liệu, sau đó sắp xếp và lấy ra top-K kết quả. Phương pháp này có độ phức tạp $O(N \cdot d)$ (với $N$ là số vector, $d$ là số chiều). Với $N$ lớn, nó trở nên quá chậm để có thể sử dụng trong thực tế.

\paragraph{Giải pháp: Approximate Nearest Neighbor (ANN) Search}
Cơ sở dữ liệu Vector không thực hiện tìm kiếm chính xác. Thay vào đó, chúng sử dụng các thuật toán \textbf{Tìm kiếm Láng giềng Gần nhất Xấp xỉ (Approximate Nearest Neighbor - ANN)}.
\begin{tcolorbox}[
    title=Triết lý của ANN,
    colback=yellow!10!white, colframe=yellow!50!black, fonttitle=\bfseries
]
Thay vì tìm kiếm các láng giềng gần nhất \textbf{chính xác tuyệt đối} với chi phí cao, chúng ta có thể tìm thấy các láng giềng \textbf{gần như chính xác} (ví dụ: 99\% đúng) với tốc độ nhanh hơn hàng nghìn lần. Sự đánh đổi nhỏ về độ chính xác này là hoàn toàn chấp nhận được trong hầu hết các ứng dụng NLP.
\end{tcolorbox}
Các thuật toán ANN như \textbf{HNSW (Hierarchical Navigable Small World)} xây dựng các cấu trúc dữ liệu giống như đồ thị thông minh trong giai đoạn lập chỉ mục, cho phép việc tìm kiếm diễn ra cực kỳ nhanh chóng bằng cách điều hướng qua đồ thị thay vì quét toàn bộ.

Một Cơ sở dữ liệu Vector là một hệ thống được tối ưu hóa để lưu trữ, lập chỉ mục, và truy vấn các vector embedding quy mô lớn bằng các thuật toán ANN.

\subsection{Các công cụ phổ biến}
\label{ssec:vector_db_tools}
\begin{itemize}
    \item \textbf{Faiss (Facebook AI Similarity Search):}
        \begin{itemize}
            \item \textbf{Bản chất:} Là một \textbf{thư viện (library)}, không phải một cơ sở dữ liệu hoàn chỉnh. Nó cung cấp các thuật toán ANN cốt lõi, hiệu năng cực cao, được viết bằng C++ và có giao diện Python.
            \item \textbf{Khi nào dùng?} Khi bạn cần hiệu năng tìm kiếm vector cao nhất có thể và sẵn sàng tự xây dựng các lớp dịch vụ (API, quản lý dữ liệu) xung quanh nó. Rất phổ biến trong môi trường nghiên cứu và các hệ thống tự xây dựng.
        \end{itemize}
    \item \textbf{ChromaDB:}
        \begin{itemize}
            \item \textbf{Bản chất:} Một cơ sở dữ liệu vector mã nguồn mở, tập trung vào sự đơn giản và dễ sử dụng. Nó có thể chạy trực tiếp trong ứng dụng Python của bạn ("in-process") mà không cần một máy chủ riêng.
            \item \textbf{Khi nào dùng?} Rất tuyệt vời cho việc phát triển nhanh, tạo mẫu (prototyping), và các ứng dụng quy mô nhỏ. Được mệnh danh là "SQLite của các vector".
        \end{itemize}
    \item \textbf{Milvus / Weaviate:}
        \begin{itemize}
            \item \textbf{Bản chất:} Là các cơ sở dữ liệu vector mã nguồn mở, đầy đủ tính năng, được thiết kế cho các hệ thống production quy mô lớn. Chúng hoạt động như các dịch vụ độc lập.
            \item \textbf{Tính năng nâng cao:} Hỗ trợ lọc metadata (ví dụ: "tìm các vector tương tự nhưng chỉ trong các tài liệu được tạo sau ngày X"), khả năng mở rộng (scaling), và độ sẵn sàng cao (high availability).
            \item \textbf{Khi nào dùng?} Khi xây dựng các ứng dụng RAG hoặc tìm kiếm ngữ nghĩa cho production.
        \end{itemize}
    \item \textbf{Pinecone:}
        \begin{itemize}
            \item \textbf{Bản chất:} Một dịch vụ cơ sở dữ liệu vector được \textbf{quản lý hoàn toàn (fully-managed)} trên đám mây (SaaS).
            \item \textbf{Lợi ích:} Bạn không cần phải lo lắng về việc thiết lập, bảo trì, hay mở rộng cơ sở hạ tầng. Nó cung cấp một API đơn giản để sử dụng và được tối ưu hóa cho hiệu năng cao và độ trễ thấp.
            \item \textbf{Khi nào dùng?} Khi bạn muốn có một giải pháp vector search mạnh mẽ mà không cần phải quản lý cơ sở hạ tầng.
        \end{itemize}
\end{itemize}