\chapter{QUY TRÌNH LÀM VIỆC VÀ CÔNG CỤ XỬ LÝ DỮ LIỆU}
\label{chap:workflow_data}

Nếu như các chương trước đã trang bị cho bạn kho "vũ khí" lý thuyết, thì chương này sẽ dạy bạn cách trở thành một "chiến binh" thực thụ trên chiến trường NLP. Một mô hình tinh vi nhất cũng sẽ trở nên vô dụng nếu không có dữ liệu chất lượng cao để học hỏi và một quy trình làm việc bài bản để định hướng.

Trong chương mở đầu của Phần 2, chúng ta sẽ tập trung vào nền móng của mọi dự án thành công: \textbf{Dữ liệu và Quy trình}. Chúng ta sẽ phác thảo một vòng đời dự án NLP tinh gọn, từ khâu lên ý tưởng, thiết lập baseline, cho đến khi triển khai và giám sát. Quan trọng hơn, chúng ta sẽ đi sâu vào các kỹ thuật và công cụ thiết yếu để thu thập, làm sạch, gán nhãn, và quản lý phiên bản dữ liệu -- những công việc chiếm đến 80\% thời gian của một dự án thực tế. Nắm vững những kỹ năng này là bước đi đầu tiên và quan trọng nhất để biến lý thuyết thành những sản phẩm có giá trị.