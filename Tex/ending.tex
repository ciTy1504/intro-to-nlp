% File: ending.tex
\chapter*{Lời kết}
\addcontentsline{toc}{chapter}{Lời kết} 
\markboth{Lời kết}{} 

\begin{center}
    \vspace*{4cm}
    \rule{0.8\textwidth}{0.4pt} \\[0.8cm]
    \parbox{0.9\textwidth}{
        \centering
        \textit{
        Thế là ta đã đi hết một vòng  Xử lý Ngôn ngữ Tự nhiên. Xin chúc mừng
        }
    } \\[1cm]

    \rule{0.8\textwidth}{0.4pt} \\[2cm]
\end{center}

\noindent
\textbf{Con đường phía trước}

\paragraph{Phần 1}  
Đừng coi “Bách khoa Toàn thư về Lý thuyết” là thứ đọc xong rồi xếp xó. Nó chính là cái nền để bạn dựng mọi mô hình sau này. 
Khi có một paper mới ra với tên nghe như thần chú, quay lại phần này: bạn sẽ thấy bản chất nó vẫn xoay quanh Attention, 
học biểu diễn, và mấy định luật xác suất quen thuộc. Muốn làm pro thật sự thì không chỉ biết xài tool, mà còn phải hiểu vì sao nó chạy được.  

\paragraph{Phần 2}  
“Cẩm nang Thực chiến” đưa công thức cho bạn, nhưng công thức thì sinh ra là để phá. 
Hãy nghịch: đổi mô hình, đổi dữ liệu, phá optimizer, thậm chí làm sai cũng được.  
Mỗi lần máy báo lỗi, hãy coi đó là một mentor thầm lặng. Không giáo trình nào dạy nhanh bằng cách tự fix một đống bug.

\paragraph{Cuối cùng, đừng quên trách nhiệm.}  
NLP không chỉ là code chạy đúng hay loss xuống đẹp. Những mô hình bạn build hôm nay có thể ảnh hưởng đến cách con người giao tiếp, tiếp nhận thông tin.
Thế nên, ngoài chuyện optimize cho nhanh, hãy luôn nghĩ đến bias, fairness, và impact. Làm kỹ sư NLP nên có trách nhiệm đi kèm.

\vfill

\begin{center}
    \parbox{0.9\textwidth}{
        \centering
        {\itshape
        Cảm ơn bạn đã đồng hành tới cuối chặng đường này.  
        Cuốn sách thì đóng lại, nhưng bug thì vẫn còn nhiều, và hành trình học của bạn mới chỉ bắt đầu thôi.  
        Chúc bạn vừa code giỏi, vừa ngủ đủ giấc!
        }
    }
\end{center}
